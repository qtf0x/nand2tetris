\documentclass[12pt]{article}

\usepackage{fancyhdr} % for header/footer control
\usepackage[margin=85pt]{geometry}

\begin{document}

\pagenumbering{gobble} % supress page numbers

\pagestyle{fancy}
\fancyhead{} % clear all header fields

\fancyhead[L]{CSCI 410}
\fancyhead[C]{TPOTS: Chapters 3 \& 4}
\fancyhead[R]{Vincent Marias}

While the first two chapters covered fundamental ideas underlying computer
hardware, in chapter 3 we jump all the way up to the level of software, with an
overview of the act of \textit{programming} a computer. Hillis compares
programming languages and their constructs to those of natural human languages,
making the crucial distinction that programming languages are \textit{formal
languages}, with strict syntactic and semantic rules, contrasting the ambiguity
inherent in how humans parse spoken language. We get an introduction to
functional abstraction via the toy programming language Logo, as well as the
treatment of program instructions as data. We finish with the briefest of
introductions to the idea of program translation, and a high-level overview of
the complete process by which a computer program is written, translated, and
executed.

Chapter 4 is mostly an introduction to concepts in computability theory. We
start with a brief explanation of Turing machines, the main idea of the section
being that all computing devices are essentially equivalent; they are all able
to perform the exact same set of computations, regardless of differential
speeds or memory sizes. There's a section about RNG, which just explains the
fundamentals (chaotic systems, true randomness, etc.), followed by a section on
about \textit{noncomputable} problems and the Church thesis. We finish with
some ideas about quantum computing that are, from what I understand, quite
outdated, though theoretically sound.

\vspace{1em}

I'm certainly not surprised by anything in chapter 3, but that doesn't mean it
isn't interesting. I'm impressed by how concisely Hillis is able to explain all
the most foundational ideas of programming. Logo seems like an excellent
language to teach these ideas to children, and the progression presented by
Hillis is logical and self-justifying. In only a few pages we cover
subroutines, functional abstraction, loops, conditionals, and even recursion.
As someone interested in education and pedagogical research, I enjoy reading
about such incredibly well-constructed pathways for teaching complex ideas to
children, some of which I didn't encounter until my second semester of
university.

Chapter 4 is also interesting. I've not taken the Theory of Computation course
here at Mines yet, so my exposure to these concepts in detail has been so far
minimal. The idea that the human brain could, in principle, be simulated by a
computer has come up before for me, but I've never considered the implication
that this means the human brain is really just a Turing machine, no different
from any other computational device. While the thought is, as Hillis suggests,
rather troubling at first, I don't know if it really ought to be. Why should
the human brain be any different from a computer? I think I agree with Hillis
that, if this is indeed the case, it doesn't diminish the value of human
thought and experience, only gives us a new way to think about and explore it's
boundaries. Finally, the other idea I found interesting from this chapter was
about random numbers. I had never really thought about the fact that you could
generate a truly random sequence of data from something as simple as a Geiger
counter, due to the laws of quantum mechanics. I'll need to factor that into my
own theories about determinism and free will, just as soon as I catch up on all
this homework...

\end{document}
